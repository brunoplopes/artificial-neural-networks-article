\documentclass[conference]{IEEEtran}
\IEEEoverridecommandlockouts
\usepackage{cite}
\usepackage{amsmath,amssymb,amsfonts}
\usepackage{algorithmic}
\usepackage{graphicx}
\usepackage{textcomp}
\usepackage{xcolor}
\usepackage[brazilian]{babel}
\usepackage[utf8]{inputenc}
\usepackage[T1]{fontenc}
\def\BibTeX{{\rm B\kern-.05em{\sc i\kern-.025em b}\kern-.08em
    T\kern-.1667em\lower.7ex\hbox{E}\kern-.125emX}}
\begin{document}

\title{Perceptron de Multiplas camadas para Diagnóstico de Câncer de Mama}

\author{\IEEEauthorblockN{1\textsuperscript{o} Anara Olimpio}
    \IEEEauthorblockA{
        \textit{anaraolimpio@ig.com.br}
    }
    \and
    \IEEEauthorblockN{2\textsuperscript{o} Bruno Lopes}
    \IEEEauthorblockA{
        \textit{bruno.lopes.ti@icloud.com}
    }
    \and
}
\maketitle

\begin{abstract}
O câncer de mama é considerado o segundo tipo de câncer mais recorrente em mulheres, perdendo somente para o câncer de pele. De acordo com dados do Instituto Nacional do Câncer (INCA), em 2014, ocorreram mais de 57 mil casos de câncer de mama no Brasil em mulheres e, embora em quantidades bem pequenas, em homens. Diante do número alto de incidências, principalmente em mulheres, existe uma grande necessidade de pesquisa sobre o assunto. Este trabalho será baseado numa proposta de utilização de rede neural artificial para obter informações mais rápidas sobre o diagnóstico do câncer de mama utilizando como base para estudo e desenvolvimento do trabalho o conjunto de dados Breast Cancer Wisconsin. Este conjunto de dados de câncer de mama foi obtido do Hospital Universidade de Wisconsin, Madison, do Dr. William H. Wolberg. Os dados foram coletados, periodicamente, entre 1989 e 1991 com a ajuda do doutor Wolberg ao relatar seus casos clínicos o que possibilitou coletar as medidas dos tumores de mama. Os dados refletem uma ordem cronológico da coleta de dados e é um conjunto de dados de classificação. Há duas classes determinadas: benignas e malignas. Este conjunto de dados possui 699 registros e dimensão de 9 e baseado nessas informações pretendemos criar um padrão de comportamento que seja ágil em predizer se o tumor é maligno ou benigno.

\end{abstract}

\begin{IEEEkeywords}
Perceptron. Artificial Neural Network. Breast Cancer Wisconsin
\end{IEEEkeywords}

\section{INTRODUÇÃO}
teste

\section{Referencial Teórico}
	%Problema, base conteitual para a solucao %
	
	\begin{figure}[htbp]
	\centerline{\includegraphics[scale=0.4]{pso_algoritmo.JPG}}
	\caption{Fluxograma-Algoritmo PSO}
	\label{fig}
	\end{figure}
	
	
\section{Avaliação de Desenpenho}
    % Metodologia, experimentos e resultados, analise dos resultados(porques) %
  
\section{Resultado e Discussão}

   
    
\section*{Conclusão}

  

\begin{thebibliography}{00}

\bibitem{b1} 

\end{thebibliography}
\end{document}

